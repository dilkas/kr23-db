\documentclass{article}
\pdfpagewidth=8.5in
\pdfpageheight=11in

\usepackage{kr}

% Use the postscript times font!
\usepackage{times}
\usepackage{soul}
\usepackage{url}
\usepackage[hidelinks]{hyperref}
\usepackage[utf8]{inputenc}
\usepackage[small]{caption}
\usepackage{graphicx}
\usepackage{amsmath}
\usepackage{amsthm}
\usepackage{booktabs}
% \usepackage{algorithm}
% \usepackage{algorithmic}
\urlstyle{same}

\usepackage{mathtools}
\usepackage{amssymb}
\usepackage{pifont}
\usepackage[linesnumbered,ruled,vlined]{algorithm2e}
\usepackage{tikz}
\usepackage[capitalize]{cleveref}
\usepackage{multirow}
\usepackage{colortbl}
\usepackage{complexity}
\usepackage{siunitx}
\usepackage{forest}
\usepackage[inline]{enumitem}

\usetikzlibrary{cd}
\usetikzlibrary{shapes}

% the following package is optional:
%\usepackage{latexsym}

% See https://www.overleaf.com/learn/latex/theorems_and_proofs
% for a nice explanation of how to define new theorems, but keep
% in mind that the amsthm package is already included in this
% template and that you must *not* alter the styling.
% \newtheorem{example}{Example}
\newtheorem{theorem}{Theorem}
\newtheorem{fact}{Fact}
\newtheorem{proposition}{Proposition}
\newtheorem{conjecture}{Conjecture}
\newtheorem{claim}{Claim}[theorem]
\theoremstyle{definition}
\newtheorem{definition}{Definition}
\newtheorem{example}{Example}

\pdfinfo{
/TemplateVersion (KR.2022.0, KR.2023.0)
}

\title{Synthesising Recursive Functions for First-Order Model Counting}

% TODO (at the very end)
% put the appendix in a separate file
% grammarly the whole thing
% read the formatting instructions and their LaTeX source (email font?)
% make sure there are no overflow errors
% clean up unnecessary things from the header (e.g., packages and definitions)
%  make sure the result table (as well as other floating objects) are on the right pages

\author{%
Paulius Dilkas$^1$\and
Vaishak Belle$^2$\\
\affiliations
$^1$National University of Singapore, Singapore, Singapore\\
$^2$University of Edinburgh, Edinburgh, UK\\
\emails
paulius.dilkas@nus.edu.sg,
vbelle@ed.ac.uk
}

\crefalias{diagram}{equation}
\crefname{diagram}{Diagram}{Diagrams}
\Crefname{diagram}{Diagram}{Diagrams}
\creflabelformat{diagram}{#2\textup{(#1)}#3}

\crefname{algocf}{Algorithm}{Algorithms}
\Crefname{algocf}{Algorithm}{Algorithms}
\crefname{condition}{Condition}{Conditions}
\Crefname{condition}{Condition}{Conditions}

\newcommand{\crefrangeconjunction}{--}
\newcommand\pfun{\mathrel{\ooalign{\hfil$\mapstochar\mkern5mu$\hfil\cr$\to$\cr}}}
\newcommand{\cmark}{\ding{51}}%
\newcommand{\xmark}{\ding{55}}%
\newcommand{\FOtwo}{$\mathsf{FO}^{2}$}
\newcommand{\FOthree}{$\mathsf{FO}^{3}$}
\newcommand{\SFO}{$\mathsf{S}^{2}\mathsf{FO}^{2}$}
\newcommand{\SRU}{$\mathsf{S}^{2}\mathsf{RU}$}
\newcommand{\Uone}{$\mathsf{U}_{1}$}
\newcommand{\Ctwo}{$\mathsf{C}^{2}$}
\newcommand{\IFO}{$\mathsf{I}\mathsf{FO}^{2}$}

\DeclareMathOperator{\friends}{\texttt{friends}}
\DeclareMathOperator{\smokes}{\texttt{smokes}}
\DeclareMathOperator{\CR}{\textsc{CR}}
\DeclareMathOperator{\GDR}{\textsc{GDR}}
\DeclareMathOperator{\Reff}{\textsc{Ref}}
\DeclareMathOperator{\dom}{dom}
\DeclareMathOperator{\Doms}{Doms}
\DeclareMathOperator{\Vars}{Vars}
\DeclareMathOperator{\Preds}{Preds}
\SetKwProg{Fn}{Function}{:}{}
\SetKwFunction{identifyRecursion}{r}
\SetKwFunction{generateMaps}{genMaps}

\makeatletter
\newcommand{\nosemic}{\renewcommand{\@endalgocfline}{\relax}}% Drop semi-colon ;
\newcommand{\dosemic}{\renewcommand{\@endalgocfline}{\algocf@endline}}% Reinstate semi-colon ;
\newcommand{\pushline}{\Indp}% Indent
\newcommand{\popline}{\Indm\dosemic}% Undent
\makeatother
\Crefname{algocf}{Algorithm}{Algorithms}
\crefname{line}{line}{lines}

\forestset{
sn edges/.style={for tree={edge={-Latex}}}
}

\begin{document}

\appendix
\section{Solutions Found by \textsc{Crane}}\label{app:solutions}

In this appendix, we list the exact function definitions produced by
\textsc{Crane} for all of the problem instances in Section~6 both before and
after algebraic simplification (excluding multiplications by one). The
correctness of all of them has been checked by identifying suitable base cases
and verifying the numerical answers across a range of domain sizes.

\begin{enumerate}
  \item $\Theta(m)$ solution for counting $\Gamma \to \Gamma$ functions:
  \[
    f(m) = {\left(-1 + \sum_{l=0}^{m} \binom{m}{l} [l < 2]\right)}^{m} = m^{m}.
  \]
  \item $\Theta(m^3 + n^3)$ solution for counting $\Gamma \to \Delta$
  surjections:
  \begin{align*}
    f(m, n) ={}& \sum_{l=0}^{m} \binom{m}{l}{(-1)}^{m-l} \sum_{k=0}^{n} \binom{n}{k} {(-1)}^{n-k}\\
               &{\left( \sum_{j=0}^{k} \binom{k}{j} [j < 2] \right)}^{l} \\
    ={}& \sum_{l=0}^{m} \binom{m}{l}{(-1)}^{m-l} \sum_{k=0}^{n} \binom{n}{k} {(-1)}^{n-k} {(k+1)}^{l}.
  \end{align*}
  \item $\Theta(m^{3})$ solution for counting $\Gamma \to \Gamma$ surjections:
  \begin{align*}
    f(m) ={}& \sum_{l=0}^{m} \binom{m}{l}{(-1)}^{m-l} \sum_{k=0}^{m} \binom{m}{k} {(-1)}^{m-k}\\
            &{\left( \sum_{j=0}^{k} \binom{k}{j} [j < 2] \right)}^{l} \\
    ={}& \sum_{l=0}^{m} \binom{m}{l}{(-1)}^{m-l} \sum_{k=0}^{m} \binom{m}{k} {(-1)}^{m-k} {(k+1)}^{l}.
  \end{align*}
  \item $\Theta(mn)$ solution for counting $\Gamma \to \Delta$ injections and
  partial injections (with different base cases):
  \begin{align*}
    f(m, n) ={}& \sum_{l=0}^m \binom{m}{l} [l<2] f(m-l, n-1)\\
    ={}& f(m, n-1) + mf(m-1, n-1).
  \end{align*}
  \item $\Theta(m^{3})$ solution for counting $\Gamma \to \Gamma$ injections:
  \begin{align*}
    f(m) ={}& \sum_{l=0}^{m} \binom{m}{l} {(-1)}^{m-l} g(m, l); \\
    g(m, l) ={}& \sum_{k=0}^{l} \binom{l}{k} [k < 2] g(m - 1, l - k)\\
    ={}& g(m - 1, l) + lg(m - 1, l - 1).
  \end{align*}
  % \item $\Theta(m^{2})$ solution for counting $\Gamma \to \Gamma$ partial
  %       injections:
  % \begin{align*}
  %   f(m) ={}& f(m - 1) + \sum_{l = 0}^{m} \binom{m}{l} \sum_{k = 0}^{m} \binom{m}{k}\\
  %           &[1 \le l < 3] [k < 2] 2^{m - l} g(m - l, m - k - 1)\\
  %   ={}& f(m - 1) + m2^{m-1}g(m-1, m-1)\\
  %   +{}& m^{2}2^{m-1}g(m-1, m-2)\\
  %   +{}& \binom{m}{2}2^{m-2}g(m-2, m-1) \\
  %   +{}& \binom{m}{2}m2^{m-2}g(m-2, m-2)\\
  %   g(i, j) ={}& \sum_{h = 0}^{j} \binom{j}{h} [h < 2] g(i - 1, j - h)\\
  %   ={}& g(i-1, j) + jg(i-1, j-1).
  %   \end{align*}
  \item $\Theta(m)$ solution for counting $\Gamma \to \Delta$ bijections:
  \[
    f(m, n) = mf(m-1, n-1).
  \]
\end{enumerate}

\end{document}
